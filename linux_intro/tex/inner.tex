\section{Introduction to Linux}

In this chapter you are about to learn about three major family Linux distributions, which are: 
 
\begin{enumerate}
    \item Red Hat Family Systems (Fedora, CentOS) 
    \item SUSE Family Systems (openSUSE)
    \item Debian Family Systems (Ubuntu, Linux Mint)
\end{enumerate}

 Due to user's need the amount of distributions will constanly grow as long as they can develop special configurations to respond to these needs. 

 % image with distro families on 3:02

 \subsection{Red Hat Family Systems}

 \textbf{Red Hat Enterpise Linux} or \textbf{RHEL} is the family that includes CentOS and Oracle Linux Fedora. Last one is significantly more software than RHEL beacause of community involved in building Fedora. Futhermore, RHEL is a commercial distribution that tests Fedora as a beta version of RHEL. 

 So key facts about RHEL are:
 \begin{enumerate}
    \item[$\circ$] Fedora serves as an upstream testing platform 
    \item[$\circ$] CentOS is a close clone of RHEL
    \item[$\circ$] Oracle Linux supports hardware platform such as Intel x86-64, ARM64, Itanium, PowerPC and IBM System z
    \item[$\circ$] It uses \textbf{RPM} package manager and \textbf{YUM} package manager for software management
    \item[$\circ$] It is used in enterprise environments, servers, and workstations 
 \end{enumerate}

\subsection{SUSE Family Systems}

\textbf{SUSE Linux Enterprise} or \textbf{SLE} is the family that includes openSUSE, relation between them is similar to RHEL and Fedora. SLE is a commercial distribution that tests openSUSE as a beta version of SLE. 

So key facts about SLE are:
\begin{enumerate}
    \item[$\circ$] SLES is upstream testing platform for openSUSE
    \item[$\circ$] It uses \textbf{RPM}-bazed package management system 
    \item[$\circ$] It includes tools like YaST for system administration and configuration
    \item[$\circ$] SLES is widely used in retail
\end{enumerate}  

