\begin{abstract}
Цель данной работы — изучить эффект Холла в полупроводниковом образце.
Для достижения этой цели измерялась зависимость ЭДС Холла от величины
внешнего магнитного поля при различных токах.
Экспериментальные данные показали согласие с теорией в пределах погрешностей.
\end{abstract}

% ==================================================
\section{Постановка задачи и теоретическое введение}

Эффект Холла представляет собой возникновение поперечного электрического
поля в проводящем образце, помещённом во внешнее магнитное поле.

Постоянная Холла определяется выражением
\begin{equation}
R_H = \frac{1}{nq},
\end{equation}
где $q$ — заряд носителей, $n$ — их концентрация.

Экспериментально постоянная Холла находится из
\begin{equation}
R_H = \frac{\Delta K}{\Delta I}\, h,
\end{equation}
где
\[
K = \frac{\Delta E_x}{\Delta B},
\]
а $h$ — толщина образца.

Проводимость образца:
\begin{equation}
\sigma = \frac{Il}{Uah},
\end{equation}

Подвижность носителей заряда:
\begin{equation}
\mu = \frac{\sigma}{nq}.
\end{equation}

% ==================================================
\section{Экспериментальная методика}

Схема экспериментальной установки приведена на рис.~\ref{fig:setup}.

Образец из легированного германия помещается в зазор электромагнита.
Ток через образец измеряется миллиамперметром, а напряжение между
контактами — цифровым вольтметром.

Паспортные погрешности приборов:
\begin{itemize}
    \item Вольтметр В7-78/1: $\pm(0.0035\% + 0.0005\%\ \text{диапазона})$
    \item Источник питания GW Instek GPR-11H30D: $\pm(0.2\% + 3\,\text{мА})$
    \item Милливольтмиллиамперметр М2020: $\pm0.2\%$
    \item Магнитометр Актаком АТЕ-8702: $\pm(5\% + 10\,\text{е.м.р.})$
\end{itemize}

\begin{figure}[h]
    \centering
    \includegraphics[width=0.75\textwidth]{../images/ustan/ustan.png}
    \caption{Схема экспериментальной установки}
    \label{fig:setup}
\end{figure}

% ==================================================
\section{Результаты}

График зависимости индукции магнитного поля от тока электромагнита представлен на Рис. \ref{fig:cal}.
\begin{figure}[h]
    \centering
    \includegraphics[width=0.5\textwidth]{../calibration.png}
    \caption{Градуировка электромагнита $B = f(I_M)$.}
    \label{fig:cal}
\end{figure}

\newpage

Зависимости ЭДС Холла от внешнего магнитного поля при различных токах
представлены на рис.~\ref{fig:hall_curves}.

\begin{figure}[h]
    \centering
    \includegraphics[width=0.6\textwidth]{../holl.png}
    \caption{Зависимость ЭДС Холла от магнитного поля при различных токах}
    \label{fig:hall_curves}
\end{figure}

По коэффициентам наклона графиков была определена постоянная Холла:
\[
R_H = (28.6 \pm 1.8)\cdot 10^{-3}\ \text{м}^3/\text{Кл}.
\]

Зависимость коэффициента $K$ от тока через образец приведена на
рис.~\ref{fig:k_vs_i}.

\begin{figure}[h]
    \centering
    \includegraphics[width=0.7\textwidth]{../current.png}
    \caption{Зависимость коэффициента $K$ от тока через образец}
    \label{fig:k_vs_i}
\end{figure}

\newpage

Рассчитанные параметры образца:

\begin{table}[h]
    \centering
    \renewcommand{\arraystretch}{1.5}
    \begin{tabular}{|c|c|c|c|}
        \hline
        $R_H \pm \Delta R_H,$ м$^3$/Кл $\cdot 10^{-3}$  & $n \pm \Delta n,$ м$^{-3} \cdot 10^{20}$ & $\sigma \pm \Delta \sigma,$ См/м  & $\mu \pm \Delta \mu, $ см$^2$/(В $\cdot$ с) \\
        \hline
        $27.94 \pm 1.8$ & $2.23 \pm 0.06$ & $17.5 \pm 0.9$ & $2393 \pm 140$ \\
        \hline
    \end{tabular}
    \caption{Результаты измерений и расчетов.}
\end{table}

% ==================================================
\section{Обсуждение результатов}

Полученные значения параметров образца по порядку величины соответствуют
табличным значениям для германия.
Отклонения объясняются неизвестным составом примесей в образце.

Экспериментальные результаты согласуются с теорией в пределах
погрешностей измерений.


% \subsection*{Семейство холловских характеристик}
% Построены зависимости ЭДС Холла от индукции поля при различных токах через образец (Рис. \ref{fig:family}).
% \begin{figure}[h]
%     \centering
%     \includegraphics[width=0.6\textwidth]{../holl.png}
%     \caption{Семейство характеристик $\mathcal{E}_x = f(B)$.}
%     \label{fig:family}
% \end{figure}

% \subsection*{Определение параметров полупроводника}
% Для каждой прямой определен наклон $K = \frac{\Delta \mathcal{E}_x}{\Delta B}$ (в мВ/Тл). Построен график зависимости коэффициента $K$ от тока $I$ (Рис. \ref{fig:slope}).

% \begin{figure}[h]
%     \centering
%     \includegraphics[width=0.6\textwidth]{../current.png}
%     \caption{Зависимость $K = f(I)$. Наклон прямой определяет постоянную Холла.}
%     \label{fig:slope}
% \end{figure}

% Угловой коэффициент этого графика $\alpha = \Delta K / \Delta I$ позволяет найти постоянную Холла:
% $$ R_H = \alpha \cdot a $$

% Вычисление удельной проводимости $\sigma$:
% $$ \sigma = \frac{L_{35} \cdot I}{U_{35} \cdot l \cdot a} $$
% Вычисление подвижности $\mu$ (во внесистемных единицах см$^2$/(В$\cdot$с)):
% $$ \mu = |R_H| \cdot \sigma $$


% \begin{table}[h]
%     \centering
%     \renewcommand{\arraystretch}{1.5}
%     \begin{tabular}{|c|c|c|c|}
%         \hline
%         $R_H \pm \Delta R_H,$ м$^3$/Кл $\cdot 10^{-3}$  & $n \pm \Delta n,$ м$^{-3} \cdot 10^{20}$ & $\sigma \pm \Delta \sigma,$ См/м  & $\mu \pm \Delta \mu, $ см$^2$/(В $\cdot$ с) \\
%         \hline
%         $27.94 \pm 1.8$ & $2.23 \pm 0.06$ & $17.5 \pm 0.9$ & $2393 \pm 140$ \\
%         \hline
%     \end{tabular}
%     \caption{Результаты измерений и расчетов.}
% \end{table}
